\documentclass[titlepage]{article}

\usepackage[utf8]{inputenc}
\usepackage[T1]{fontenc}
\usepackage[french]{babel}
\usepackage{mathtools, amssymb, amsthm, amsmath}
\usepackage{lmodern}
\usepackage{listings}
\usepackage{hyperref}

\newtheorem{Prop}{Propriété}
\newtheorem{proofi}{Preuve}
\newtheorem{defi}{Définition}
\newtheorem{theo1}{Lemme}
\newtheorem{theo}{Théorême}
\newtheorem{Absurde}{Demonstration par l'Absurde}
\newtheorem{dem}{Demonstration}

\newcommand{\encadrement}[1]{\noindent\fbox{
\begin{minipage}{\textwidth}
#1
\end{minipage}}}

\title{La décomposition en produit de facteurs premiers}
\author{Guillem Audebert - Florian Berger - Lyse Rousseau}
\date{\today}


\begin{document}
\maketitle
\tableofcontents

\bigskip
\section{Definition de l'ensemble des entiers naturels}


Afin d'introduire ce cours, il convient de poser l'ensemble des entiers naturels positif, que l'on note $\mathbb{N}$.

\begin{equation*}
    \mathbb{N} = \{ 0;1;2;3;... \}
\end{equation*}

Par exemple $4 \in \mathbb{N}$ mais $\pi \notin \mathbb{N}$ ou $ -3 \notin \mathbb{N}$
\par
\textit{Nombres pairs} : Tous les nombres pouvant s'écrire sous la forme $2k$ avec $k \in 
\mathbb{N}$\label{divis}.
\par
\textit{Nombres impairs} : Tous nombres pouvant s'écrire sous la forme $2k+1$ avec $k \in \mathbb{N}$.

\section{Nombres Premiers}

\encadrement{\begin{Prop}
Un entier $n \in \mathbb{N}$ est premier si et seulement si il est divisible par $n$ et par $1$.
\end{Prop}}
\medskip
\par
Avec cette définition, on peut déterminer que $17$ est un nombre premier, car $1|17$ et $17|17$, et $17$ n'admet pas d'autre diviseur.
\par
Mais $1$ n'est pas un nombre premier, car il ne satisfait pas les deux conditions néccessaires. En effet, $1$ ne possède qu'un seul diviseur : lui même : $1|1$.
\medskip

\encadrement{\begin{Prop}
L'entier $2$ est le seul nombre premier pair.
\end{Prop}}
\medskip
\par

\begin{Absurde}
Supposons $m$ un entier premier et pair différent de 2.
	\begin{align*}
    	& m = 2k 
    		& \text{avec }  k \in \mathbb{N} \\
    	& \frac{m}{2} = k
	\end{align*}

car m est pair et 2 divise m, d'après la définition des nombres pairs donnée dans la section \ref{divis}.

Donc m possède au moins trois diviseurs : \{ 1; 2; m \} 

Or ceci est absurde puisque un nombre premier ne peut avoir trois diviseurs. 

Donc m n'est pas premier.

\end{Absurde}

\encadrement{\begin{Prop}
$\forall$ $n\in\mathbb{N}$, n > 1 et n non premier, on admet au moins un diviseur premier d tel que $2 \leq  d \leq \sqrt{n}$
\end{Prop}}
\par

\begin{proofi}
En admettant que $n$ soit un entier non premier, et $E$ l'ensemble des diviseurs de $n$ en excluant $1$ et lui-même. On induit que $E$ n’est pas un ensemble vide : il contient des diviseurs de $n$, noté $d$, entre $2$ et $n-1$, tel que $E$ admet un plus petit élément $d$. \\ 
Ainsi, $d$ est premier, sinon il admettrait un diviseur autre que $1$ et lui-même. On peut alors écrire que $n = d \times p $ avec $d \leq p$, $p\in
\mathbb{N}$, car $d$ est le plus petit élément de $E$. Donc, en multipliant l’inégalité par $d$, on a : $d \times d \leq p \times d$,  soit $(d^2) \leq (p \times d = n)$, et ainsi $d \leq \sqrt{n} $; d’où $2 \leq  d \leq \sqrt{n}$.
\end{proofi}
\medskip

\encadrement{\begin{Prop}
Il existe une infinité de nombres premiers.
\end{Prop}}

\begin{proofi}
On suppose qu’il existe un nombre fini de nombres premiers $d_{1},d_{2}, …, d_{n}$. On le notera $f$.
En effet $2 \leq f + 1$, donc $f$ contient au moins un 
diviseur premier $d_{i}$ parmi $f$ soit $d_{1},d_{2}, …, d_{n}$
Alors $d_{i}$ divise $f+1$ et $f$, donc il divise leur différence, 
soit $d_{i}$ divise $1$ : $(f+1) - f =1$, ce qui est contradictoire. Il existe 
donc une infinité de nombres 
premiers.
\end{proofi}

\section{Théorèmes fondamentales de l'arithmétique}

\encadrement{\begin{theo1}
Tout nombres entier naturel $n$ strictement supérieur à 1 peut s'écrire comme un 
produit de facteur premier comme suit (On appelle cela une décomposition en 
facteurs premiers):

	\begin{equation*}
    	n = {q_{1}}^{p_{1}}\times {q_{2}}^{p_{2}}\times ...\times {q_{n}}^{p_{n}}
	\end{equation*}

 avec $q_{k}$ un nombre premier et $p_{k}$ un entier positif. \par
 On peut donner l'exemple suivant:
 
	\begin{equation*}
    	12 = 4 \times 3 = 2^{2} \times 3^{1}
	\end{equation*}
\end{theo1}}
\begin{dem}
Supposons que pour tout $n > 2$, possède un diviseur premier $p$. Et si $p$ est premier alors $p \rightarrow p|a$ ou $p|b$ avec $a=1$ et  $b=p$ ou $a=$ et $b=1$
\\ \\
On note $P_{(n)}$, l'existence d'une décomposition en facteur premier pour l'entier n.
\\ \\
\Large Initialisation : \normalsize Pour $n=2$ l'existence est vérifiée , en effet $2^{1}$ est nombre premier. Donc $P_{(2)}$ est vraie.
\\ \\ \Large Hérédité : \normalsize Montrons l'existence d'une décomposition pour $n+1$ et donc que $P_{(n+1)}$ est vraie.
\\ Selon la première proposition $n+1$ possède un diviseur premier noté $p$
\\ Donc $n+1= p \times q$ avec $q < n+1$
\\ \\ Si $q=1$ alors $n+1= q$ est une décomposition de $n+1$
\\ Sinon, selon l’hypothèse de récurrence pour $q \leq n$ , $q$  possède une décomposition, on en déduit une décomposition de $n+1$ .
\\ \\ \Large Conclusion : \normalsize Dans les deux cas $P_{(n+1)}$ est vraie. D'après le principe de récurrence $P_{(n)}$ est vraie pour tout entier $n > 2$
\end{dem}


\encadrement{\begin{theo1}
Cette décomposition est unique pour chaque  nombre: On dit qu'il y a unicité de 
la décomposition.
\end{theo1}}

\begin{dem}
Supposons $m$ un nombre pouvant s'écrire de deux manières différentes: 
	\begin{equation*}
		m = {q_{1}}^{a_{1}}\times {q_{2}}^{a_{2}}\times ...
		\times {q_{n}}^{a_{n}} = {p_{1}}^{b_{1}}\times {p_{2}}^{b_{2}}
		\times ...\times {p_{n}}^{b_{n}}
	\end{equation*}

$q_{1}$ divise $m$, donc $q_{1}$ divise $p_{1} \times p_{2} \times p_{3} \times 
... \times p_{m}$ \par \medskip
D'après le lemme d'Euclide, soit $q_{1}$ divise $p_{1}$ (c'est-à-dire, 
nécessairement, est égal à $p_{1}$), soit il divise $p_{2} \times p_{3} \times 
... \times p_{m}$. En recommençant avec $q_{1}$ et $p_{2} \times p_{3} \times ... 
\times p_{m}$ on en déduit que $q_{1}$ est égal à l'un des facteurs $p_{k}$. Sans perdre de généralité, disons que $q_{1} = p_{1}$. On simplifie par $q_{1}$ et on recommence sur

$q_{2} \times q_{3} \times ... \times q_{n} = p_{2} \times p_{3} \times ... 
\times q_{m}$\bigskip

Finalement on arrive forcément à $n = m$ (sinon il y aurait une contradiction). 
Et chaque $p_{i}$ correspond à un et un seul $q_{i}$. Les deux décompositions sont identiques.
\end{dem}

\section{Exemple de factorisation en facteurs premiers :}
\begin{dem}
Voici quelques nombres premiers : 2, 3, 5, 7, 11, 13, 17, 19...
On va dans un odre croissant prendre chacun des nombres premiers pour oir si ce 
dernier divise bien ce nombre petit à petit dans une division euclidienne 
successive jusqu'à que sont reste soit nul.\par
\begin{tabular}{r|r c l|l}
$84$ &  &  & $36$& \\
\hline
$84$ & $2 $&   &$ 36 $&$ 2$\\
$42 $& $2 $&   & $18 $& $2$\\
$21$ & $3 $&   & $9 $& $3$\\
$7 $& $7 $&   & $3 $& $3$\\
$1$ & $0$ &   & $1$ & $0$\\
\end{tabular}
\end{dem}
Soit : \par
$84 = 2^{2} \times 3 \times 7 $\par
$36 = 2^{2} \times 3^{2}$

\section{PGCD et PPCM}

\subsection{PGCD :}

\encadrement{\begin{defi} 
Soit a et b deux entiers relatifs non tous nuls.\\
L’ensemble des diviseurs communs à a et b admet un plus grand élément d,
appelé plus grand commun diviseur, que l’on note PGCD(a;b).
\end{defi}}

\begin{Prop}
$PGCD (a,b) = PGCD(b,a)$\\
$PGCD (a,b) = PGCD(|a|,|b|)$\\
$PGCD(a,0) = a$\\
$PGCD(ka,kb) = k \times PGCD(a,b)$  avec $k$ un entier naturel non nul.
\end{Prop}

\encadrement{\begin{defi} 
Soit a et b deux entiers relatifs non nuls.\\
a et b sont premiers entre eux si $PGCD(a;b) = 1$.
\end{defi}}

\subsection{PPCM :}

\encadrement{\begin{defi}
Soit a et b deux entiers relatifs non nuls.\\
L’ensemble des multiples strictement positifs communs à a et b admet un plus petit élément que l’on note PPCM(a;b).
\end{defi}}

\begin{Prop}
Soit a et b premiers entre eux,\\
$PPCM(a ; b) = a \times b$\\
$PPCM(ka ; kb) = k \times PPCM(a ; b)$; avec $k \in {\mathbb{N}}^{*}$
\end{Prop}

\encadrement{\begin{Prop}
Soit a et b deux entiers naturels non nuls.\\
L’ensemble des multiples de PPCM(a ; b) est contenu dans l'ensemble des multiples communs de a et de b.
\end{Prop}}

\begin{proofi} 
Soit m un multiple du PPCM(a ; b),\\
Ainsi m est multiple à a et à b.\\
En effet, l'ensemble des multiples communs à a et à b est l'ensemble des multiples de leur PPCM.
\end{proofi}

\begin{theo}
 Nous pouvons établir avec le pgcd(a,b) et le ppcm(a,b)  la relation suivante:  

	\begin{align*}
	\frac{ppcm(a,b)}{pgcd(a,b)} = a \times b
	\end{align*}
	
\end{theo}

\subsection{Lien PGCD et PPCM décomposition en facteurs 
premiers}
\begin{Prop}
Soit a et b deux entiers non nuls. \\
a et b sont dit premiers entre eux si et seulement si PGCD(a;b)=1.
\end{Prop}

\begin{Prop}
Soit a et b deux entiers naturels différents de 0 et 1, qui ne sont pas premiers, et se décomposent en produit de facteurs premiers comme suit : 

	\begin{align*}
	a & = {P_{1}}^{f_{1}} \times {P_{2}}^{f_{2}} \times {P_{3}}^{f_{3}} … 
\times  {P_{n}}^{f_{n}}\\
	b & = {P_{1}}^{F_{1}} \times {P_{2}}^{F_{2}} \times {P_{3}}^{F_{3}} … 
\times  {P_{n}}^{F_{n}}\\
	\end{align*}

\noindent où $P_{1},P_{2},P_{3} …, P_{1}$ sont des nombres premiers,\\
et $f_{1},f_{2},f_{3} …, f_{n}$, $F_{1},F_{2},F_{3} …, F_{n}$, des entiers naturels éventuellement nuls.\\
Pour toute valeur k comprise entre 1 et n, on pose $m_{k}$ comme le minimum entre $f_{k}$ et $F_{k}$, et $M_{k}$ comme le maximum entre $f_{k}$ et $F_{k}$.\\
D’où $PGCD(a,b) = {P_{1}}^{m_{1}} \times {P_{2}}^{m_{2}} \times {P_{3}}^{m_{3}} … \times  {P_{n}}^{m_{n}}$\\
et $PPCM(a,b) = {P_{1}}^{M_{1}} \times {P_{2}}^{M_{2}} \times {P_{3}}^{M_{3}} … \times  {P_{n}}^{M_{n}}$\\
\end{Prop}

\section{Décomposition en produit de facteurs premiers : Algorithme}

Nous avons développé un algorithme permettant la décomposition d'un entier sous forme de produit de facteurs premiers. Le programme, sous Python, est le suivant :
\medskip

\lstinputlisting[language=python]{Decomposition_facteurs_premiers.py}

\medskip \par L'algorithme s'exécute comme suit :

\begin{enumerate}
\item \underline{La fonction "liste\_facteurs" :}
\medskip
 
	\begin{itemize}
 	\item La fonction prend en argument un entier.
 	\item Une liste "facteurs" est créée, et sera complétée par l'ensemble des facteurs premiers de l'entier en argument.
 	\item "variable" prend la valeur de "entier", et "diviseur" prend pour valeur "2" (le premier entier premier).
 	\item On vérifie que la variable est divisible par 2. Si c'est le cas, 2 est ajouté à la liste des facteurs premiers.
 	\item "variable" prend pour valeur le quotient de "variable" par 2.
 	\item On répète l'opération tant que 2 divise "variable".
 	\item Dès que "variable" n'est plus divisible par 2, on incrémente la variable "diviseur" par +1. 
 	\item Les nombres non premiers seront ainsi ignorés, car ils sont, par définition, des produits de facteurs premiers inférieurs à eux, et seront donc "décomposés" avant de pouvoir être employés comme diviseur.
 \end{itemize}
 
\medskip
\item \underline{La fonction "occurrence\_liste" :}
\medskip

	\begin{itemize}
	\item La fonction prend en argument une liste donnée.
	\item En important la fonction "Counter" du paquet "collections", on obtient un dictionnaire faisant apparaître le nombre de fois qu'un élément apparaît dans une liste.
	\end{itemize}

\medskip
\item \underline{La fonction "decomposition" :}
\medskip 

	\begin{itemize}
	\item La fonction prend en argument un entier.
	\item En employant les fonctions "liste\_facteurs" puis "occurrence\_liste" susmentionnées, on obtient un dictionnaire faisant apparaître les facteurs premiers de l'entier choisi, ainsi que leur nombre d'apparition (ce nombre sera la puissance du facteur premier).
	\item La fonction met ensuite en page les différents éléments pour faire apparaître la décomposition en produit de facteurs premiers.
	\end{itemize}
\end{enumerate}



   
\section{Bibliographie}
\begin{enumerate}
\item \href{http://pierrelux.net/documents/cours/ts/premiers_ppcm.pdf}{Pierrelux : "Les Nombres Premiers - PPCM"}
\item \href{http://licence-math.univ-lyon1.fr/lib/exe/fetch.php?media=p13:algii:1-arith.pdf}{Université Grenoble I, Didieur Piau, Bernard Ycart : "Arithmétique"}
\item \href{https://www.lyceedadultes.fr/sitepedagogique/documents/math/mathCRPEcours/09_crpe_nombres_premiers_pgcd_et_ppcm.pdf}{Lycée adulte : "Nombres Premiers. PGCD et PPCM"}
\item \href{http://math.univ-lyon1.fr/capes/IMG/pdf/new.pgcd.pdf}{Lyon 1 : "PGCD et PPCM. Nombres premiers entre eux"}
\item \href{http://math.univ-lyon1.fr/capes/IMG/pdf/new.premier.pdf}{Lyon 1 : "Les nombres premiers"}
\item \href{https://www.lyceedadultes.fr/sitepedagogique/documents/math/mathTermSspe/03_Nombres_premiers/03_cours_les_nombres_premiers.pdf}{Lycee adulte : "Les nombres premiers"}
\end{enumerate}

\end{document}